\section{INTRODUÇÃO}\label{sec:introducao}
Autômatos Celulares (ACs) são sistemas dinâmicos discretos em tempo e espaço cuja dinâmica tem sido extensivamente estudada e aplicada em diversas áreas. ACs tem a capacidade de através de regras de comportamentos locais simples, gerar comportamentos globais complexos. De acordo com \citeonline{wolfram1994cellular}, ACs podem também ser considerados uma idealização discreta de equações parciais diferenciais, muitas vezes utilizadas para descrever sistemas naturais.

Existem diversas famílias de autômatos celulares que podem ser estudadas. Devido ao rápido crescimento das famílias dos ACs conforme se mudam seu parâmetros de raio e estado, uma das famílias mais estudadas é a do espaço elementar, por possuir apenas 256 regras.

Há casos em que os estudos de ACs concentram-se em algum comportamento obtido através de restrições aplicadas às tabelas de transição. Os ACs confinados, criados por \citeonline{theyssier2004captive}, é um dos caso que se pode usar como exemplo. Esses comportamentos e propriedades obtidos através de restrições aplicadas à tabela de transição podem ser denominados como propriedade estática.

Propriedades estáticas permitem prever determinados comportamentos de um AC sem consultar sua evolução espaço-temporal, ou seja, dispensando a simulação do sistema. Propriedades estáticas também podem ser descritas como indicadores de comportamento de uma determinada família de ACs. Um exemplo de propriedade estática, descrita posteriormente em mais detalhes, é a conservabilidade de paridade. A conservabilidade de paridade define um tipo de AC binário que mantêm o número de estados com valor $1$ sempre com a mesma paridade.

%TODO: Verificar se uso referencia ao artigo do PP, ou à dissertação do Verardo
Existem algumas formas de representar propriedades estáticas, e essa representação é crucial pois culmina na eliminação da necessidade de se buscar uma propriedade analisando todo o espaço de um AC. Em \citeonline{li1990structure} já foi introduzido variáveis na representação de conjuntos de tabelas de transição. Por meio de grafos de De Bruijn, \citeonline{Betel2013} representa um conjunto de ACs na busca pela solução do problema de paridade. Mais recentemente \citeonline{deOliveira2014} estabeleceu uma representação formal para conjuntos de ACs, denominada \textit{Templates}, assim como esse estudo exemplificou o uso da biblioteca \textit{CATemplates} \cite{CATemplates} desenvolvida na linguagem do software \textit{Wolfram Mathematica} \cite{woframMathematica10}.

Templates tem a capacidade de representar conjuntos de ACs sem a necessidade de se operar uma busca em todo espaço original do AC. Essa capacidade dos templates é o principal motivador deste trabalho, visto que essa habilidade é muito importante para a resolução de diversos problemas que buscam por regras com algum comportamento específico.

\subsection{Objetivos}
Este estudo tem como objetivos principais desenvolver novos algoritmos geradores de templates baseado em propriedades estática, assim como apresentar o funcionamento da operação de complemento de templates. Ademais esse projeto propõem-se a apresentar exemplos da utilidade de templates em problemas típicos de autômatos celulares, como o problema de paridade e o problema de densidade.

Até o presente momento os seguintes itens já foram concluídos:
      \begin{itemize}
          \item Implementação da operação de complemento de templates em autômatos celulares binários.
          \item Apresentação de um conjunto de processos que restringem do espaço de busca para a solução do problema de paridade, exemplificando a utilidade de templates em problemas típicos de autômatos celulares.
      \end{itemize}

\subsection{Organização do Documento}
Este documento está organizado da seguinte forma: na Seção \ref{sec:acs} são detalhados os ACs, assim como alguma de suas propriedades. A Seção \ref{sec:templates} apresenta em mais detalhes o funcionamento dos templates, assim como descreve o funcionamentos de duas de suas principais operações. A Seção \ref{sec:propriedadesEstaticas} apresenta algumas propriedade estáticas e explica o funcionamento de suas respectivas operações de geração de template. A Seção \ref{sec:aplicacao} apresenta alguns exemplos de aplicações para os templates, focado principalmente no problema de paridade. Por fim, a Seção \ref{sec:conclusao} apresenta as considerações finais do presente trabalho.
